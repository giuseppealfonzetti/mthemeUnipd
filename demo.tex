\documentclass[aspectratio=169, 15pt]{beamer}
\usepackage{appendixnumberbeamer}
\usepackage{booktabs}
\usepackage[scale=2]{ccicons}
\usepackage{hyperref}
\usepackage{color, colortbl}
\usepackage{xcolor}
\usepackage{graphicx}
\usepackage{lmodern}
\usepackage{booktabs}
\usepackage{multirow}
\usepackage{graphicx}

\usepackage{tabularray}
\UseTblrLibrary{booktabs}

\graphicspath{{figures/}}
\setbeamercolor{math text}{fg=black}
\setbeamercolor{math text displayed}{fg=black}
\setbeamertemplate{footnote}{%
  \parindent 1em\noindent%
  \raggedright
  \insertfootnotetext\par%
}

\usepackage{subcaption}
\captionsetup{font=scriptsize}
\usepackage[backend = bibtex, style = verbose-ibid]{biblatex}
\usepackage{pgfplots}
\usepgfplotslibrary{dateplot}

\usepackage{xspace}
\newcommand{\metro}{\textbf{\textsc{metropolis}}\xspace}
\newcommand{\pres}{\textbf{\textsc{presento}}\xspace}

\addbibresource{references.bib}
\usetheme[progressbar=frametitle,subsectionpage=progressbar]{metropolis}
% custom colors
\definecolor{unipdred}{RGB}{155, 0, 20}
\definecolor{lightgray}{gray}{0.9}
\definecolor{darkgray}{gray}{0.8}
\definecolor{examplecolor}{RGB}{13, 71, 161}

%define template colors
\setbeamercolor{frametitle}{bg=unipdred}
\setbeamercolor{progress bar}{fg = unipdred}
\setbeamercolor{alerted text}{fg = unipdred}
\setbeamercolor{example text}{fg = examplecolor}

%set links color
\hypersetup{
	citecolor={blue}
}
%bibliography style
%\bibliographystyle{apalike}

%custom version of page counter from presento theme
\newcounter{totalfr}
\setbeamertemplate{footline}{
  \ifnum\inserttotalframenumber=1
    \setcounter{totalfr}{2}
  \else
     \setcounter{totalfr}{\inserttotalframenumber}
  \fi
  \hfill{
    \tikz{
      \filldraw[fill=darkgray!40, draw=darkgray!50]  (0,0) -- (0.2,0) arc (0:{\value{framenumber}*(-360/(\value{totalfr}-1))}:0.2) -- (0,0);
      \node at (0,0) {\normalsize \color{unipdred}\footnotesize{\insertframenumber}};
    }
  }
  \hspace{2em}
  \vspace*{1em}
}



\addtobeamertemplate{block alerted begin}{%
  \setlength{\textwidth}{1\textwidth}%
}{}
\definecolor{codegray}{gray}{0.9}
\newcommand{\code}[1]{\colorbox{codegray}{\text{#1}}}
\title{Title of the presentation}
\subtitle{Subtitle}
\date{}
\author{Giuseppe Alfonzetti}
\institute{University of Padova}
\titlegraphic{%
  \begin{picture}(0,0)
    \put(400,-180){\makebox(0,0)[rt]{\includegraphics[width=3cm]{logo_unipd.png}}}
  \end{picture}}
\begin{document}
\maketitle
%%%%%
\begin{frame}{Table of contents}
\vspace{0.2cm}
\footnotesize
  \setbeamertemplate{section in toc}[sections numbered]
  \tableofcontents
\end{frame}
%%%%%
\section{Introduction}
\begin{frame}[fragile]{Template}

This Beamer template uses a customised version of the \metro\footnote{For a detailed documentation see: \url{https://mirrors.ibiblio.org/CTAN/macros/latex/contrib/beamer-contrib/themes/metropolis/doc/metropolistheme.pdf}.} Beamer theme with page counter from \pres:

\begin{description}
\item[\alert{\metro}] \url{https://github.com/matze/mtheme};
\item[\alert{\pres}] \url{https://github.com/RatulSaha/presento};
\end{description}
and main theme color from:
\begin{description}
\item[\alert{Unipd}] \url{https://www.unipd.it/sites/unipd.it/files/2017/Manuale_Corporate_sigillo.pdf}.
\end{description}
\end{frame}
%%%%%
\section{Metropolis demo}

\begin{frame}[fragile]{Typography}
      \begin{verbatim}The theme provides sensible defaults to
\emph{emphasize} text, \alert{accent} parts
or show \textbf{bold} results.
\end{verbatim}

  \begin{center}becomes\end{center}

  The theme provides sensible defaults to \emph{emphasize} text,
  \alert{accent} parts or show \textbf{bold} results.
\end{frame}

\begin{frame}{Font feature test}
  \begin{itemize}
    \item Regular
    \item \textit{Italic}
    \item \textsc{Small Caps}
    \item \textbf{Bold}
    \item \textbf{\textit{Bold Italic}}
    \item \textbf{\textsc{Bold Small Caps}}
    \item \texttt{Monospace}
    \item \texttt{\textit{Monospace Italic}}
    \item \texttt{\textbf{Monospace Bold}}
    \item \texttt{\textbf{\textit{Monospace Bold Italic}}}
  \end{itemize}
\end{frame}

\begin{frame}{Lists}
  \begin{columns}[T,onlytextwidth]
    \column{0.33\textwidth}
      Items
      \begin{itemize}
        \item Milk \item Eggs \item Potatoes
      \end{itemize}

    \column{0.33\textwidth}
      Enumerations
      \begin{enumerate}
        \item First, \item Second and \item Last.
      \end{enumerate}

    \column{0.33\textwidth}
      Descriptions
      \begin{description}
        \item[PowerPoint] Meeh. \item[Beamer] Yeeeha.
      \end{description}
  \end{columns}
\end{frame}
\begin{frame}{Animation}
  \begin{itemize}[<+- | alert@+>]
    \item \alert<4>{This is\only<4>{ really} important}
    \item Now this
    \item And now this
  \end{itemize}
\end{frame}
\begin{frame}{Figures}
  \begin{figure}
    \newcounter{density}
    \setcounter{density}{20}
    \begin{tikzpicture}
      \def\couleur{alerted text.fg}
      \path[coordinate] (0,0)  coordinate(A)
                  ++( 90:5cm) coordinate(B)
                  ++(0:5cm) coordinate(C)
                  ++(-90:5cm) coordinate(D);
      \draw[fill=\couleur!\thedensity] (A) -- (B) -- (C) --(D) -- cycle;
      \foreach \x in {1,...,40}{%
          \pgfmathsetcounter{density}{\thedensity+20}
          \setcounter{density}{\thedensity}
          \path[coordinate] coordinate(X) at (A){};
          \path[coordinate] (A) -- (B) coordinate[pos=.10](A)
                              -- (C) coordinate[pos=.10](B)
                              -- (D) coordinate[pos=.10](C)
                              -- (X) coordinate[pos=.10](D);
          \draw[fill=\couleur!\thedensity] (A)--(B)--(C)-- (D) -- cycle;
      }
    \end{tikzpicture}
    \caption{Rotated square from
    \href{http://www.texample.net/tikz/examples/rotated-polygons/}{texample.net}.}
  \end{figure}
\end{frame}
\begin{frame}{Tables}
  \begin{table}
    \caption{Largest cities in the world (source: Wikipedia)}
    \begin{tabular}{@{} lr @{}}
      \toprule
      City & Population\\
      \midrule
      Mexico City & 20,116,842\\
      Shanghai & 19,210,000\\
      Peking & 15,796,450\\
      Istanbul & 14,160,467\\
      \bottomrule
    \end{tabular}
  \end{table}
\end{frame}
\begin{frame}{Blocks}
  Three different block environments are pre-defined and may be styled with an
  optional background color.

  \begin{columns}[T,onlytextwidth]
    \column{0.5\textwidth}
      \begin{block}{Default}
        Block content.
      \end{block}

      \begin{alertblock}{Alert}
        Block content.
      \end{alertblock}

      \begin{exampleblock}{Example}
        Block content.
      \end{exampleblock}

    \column{0.5\textwidth}

      \metroset{block=fill}

      \begin{block}{Default}
        Block content.
      \end{block}

      \begin{alertblock}{Alert}
        Block content.
      \end{alertblock}

      \begin{exampleblock}{Example}
        Block content.
      \end{exampleblock}

  \end{columns}
\end{frame}
\begin{frame}{Math}
  \begin{equation*}
    e = \lim_{n\to \infty} \left(1 + \frac{1}{n}\right)^n
  \end{equation*}
\end{frame}
\begin{frame}{Line plots}
  \begin{figure}
    \begin{tikzpicture}
      \begin{axis}[
        mlineplot,
        width=0.9\textwidth,
        height=6cm,
      ]

        \addplot {sin(deg(x))};
        \addplot+[samples=100] {sin(deg(2*x))};

      \end{axis}
    \end{tikzpicture}
          \caption{Caption}
  \end{figure}
  
\end{frame}
\begin{frame}{Bar charts}
  \begin{figure}
    \begin{tikzpicture}
      \begin{axis}[
        mbarplot,
        xlabel={},
        ylabel={Bar},
        width=0.9\textwidth,
        height=6cm,
      ]

      \addplot plot coordinates {(1, 20) (2, 25) (3, 22.4) (4, 12.4)};
      \addplot plot coordinates {(1, 18) (2, 24) (3, 23.5) (4, 13.2)};
      \addplot plot coordinates {(1, 10) (2, 19) (3, 25) (4, 15.2)};

      \legend{lorem, ipsum, dolor}

      \end{axis}
    \end{tikzpicture}
      \caption{Caption}
  \end{figure}
\end{frame}
\begin{frame}{Quotes}
  \begin{quote}
    Veni, Vidi, Vici
  \end{quote}
\end{frame}

{%
\setbeamertemplate{frame footer}{My custom footer}
\begin{frame}[fragile]{Frame footer}
    \metro defines a custom beamer template to add a text to the footer. It can be set via
    \begin{verbatim}\setbeamertemplate{frame footer}{My custom footer}\end{verbatim}
\end{frame}
}

\begin{frame}{References}
  Some references to showcase: one\footcite{knuth92},  two\footcite{ConcreteMath}, and three\footcite{Er01}.
\end{frame}

\section{Conclusion}

\begin{frame}{Summary}

  Get the template of this demo presentation at

  \begin{center}\url{https://github.com/giuseppealfonzetti/mthemeUnipd}\end{center}

  and feel free to adapt it to your needs.

\end{frame}
%%%%%%%%%%%%%%%%%%%%%%%%%%%%%%%%%%%%%%%%%%%%%%%%%%%%%%%%%%
\begin{frame}[allowframebreaks]
        \frametitle{References}
        \footnotesize
    \printbibliography[heading=none]
\end{frame}
\appendix
{\setbeamercolor{palette primary}{fg=white, bg=unipdred}
\begin{frame}[standout]
  Thank you for your attention!
\end{frame}
}
\begin{frame}[fragile]{Backup slides}
  Sometimes, it is useful to add slides at the end of your presentation to
  refer to during audience questions.

  The best way to do this is to include the \verb|appendixnumberbeamer|
  package in your preamble and call \verb|\appendix| before your backup slides.

  \metro will automatically turn off slide numbering and progress bars for
  slides in the appendix.
\end{frame}
\end{document}
